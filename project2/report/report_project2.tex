\documentclass{article}
\title{Project 2 \\ FYS-STK3155/4155}
\author{Magnus Bergkvam}
\usepackage{url}
\usepackage{amsmath}
\usepackage{graphicx}
\usepackage{subfig}
\usepackage{bm}


\begin{document}
\maketitle
\bibliographystyle{unsrt}

\section{Abstract}
We tested out performances of neural networks for regression and binary
classification purposes, and compared this with the models we explored in the
previous project \cite{githubrepoproject1}, as well as our own implementation of
logistic regression. We here implemented our own neural network from scratch,
basing the training on the backpropagation-algorithm. We also tested out various
different optimization techniques for optimizing the loss, including stochastic
gradient descent and the usage of different schedulers. We tested out regression
on the franke-function which we also studied in project 1
\cite{githubrepoproject1}, while for classification we used breast cancer data
from wisconsin \cite{sklearncancerdata} \cite{breastcancerwisconsin}. What we
saw was that for the franke-function with regularization we did not get any
major performance gains compared to the simpler linear models we tested in
project 1 \cite{githubrepoproject1}. For the cancer data we also saw that while
the neural networks did perform well, logistic regression with some
regularization also here performed comparably. Throughout the project we have
manifested again that just because neural networks might be powerful and
impressive in and of themselves, they are not necessarily the best option
always. We also saw how things like hidden layer sizes and activation functions
didn't matter all that much for the end performance, while regularization and
the learning-rate/learning-rate-scheduler mattered a lot for the performance in
the end.

\section{Introduction}

\section{Methods}
\subsection{The Newton-Raphson method}
Very central to the numerical optimization we do is the Newton-Raphson method.
This is a method which allows us to minimize some function by using the gradient
of the function. The method has it's origins in the Taylor-expansion of the
function we wish to minimize. In our case we use this to minimize cost-functions
for the models we use. Let $C(\bm{\beta}; \bm{X}, \bm{y})$ be some general
cost-function depending on some design matrix $\bm{X}$ and a response vector
$\bm{y}$. We wish to find the $\bm{\beta}$ which minimizes this cost, given
$\bm{X}$ and $\bm{y}$, i.e. we minimize $C$ with respect to $\bm{\beta}$, while
keeping $\bm{X}$ and $\bm{y}$ fixed. We can then approximate the cost using a
Taylor-expansion of second order around some value $\bm{\beta}^*$, in which case
we get:
\begin{align*}
    C(\bm{\beta}, \bm{X}, \bm{y}) & \approx C(\bm{\beta}^*; \bm{X}, \bm{y}) + \frac{\partial C(\bm{\beta}^*; \bm{X}, \bm{y})}{\partial \beta}(\beta - \bm{\beta}^*) + \frac{1}{2} \frac{\partial^2 C(\bm{\beta}; \bm{X}, \bm{y})}{\partial^2 \bm{\beta}}(\bm{\beta} - \bm{\beta}^*)^2 \\
                                  & = C(\bm{\beta}^*; \bm{X}, \bm{y}) + \nabla_{\bm{\beta}} C(\bm{\beta}^*) (\beta - \bm{\beta}^*) + \frac{1}{2} \bm{H}_{\bm{\beta}}(\bm{\beta}^*) (\bm{\beta} - \bm{\beta}^*)^2
\end{align*}
Where $\nabla_{\bm{\beta}} C$ is the gradient of $C$ w.r.t. $\bm{\beta}$ and
$\bm{H}_{\bm{\beta}}$ is the Hesse-matrix (second derivative) of $C$ w.r.t.
$\bm{\beta}$. This should be a good approximation for $C$ sufficiently close to
$\bm{\beta}^*$. We then can find the derivative of this w.r.t. $\bm{\beta}$:
\begin{align*}
    \frac{\partial C(\bm{\beta}, \bm{X}, \bm{y})}{\partial \bm{\beta}} & \approx
    \nabla_{\bm{\beta}} C(\bm{\beta}^*) + \bm{H}_{\bm{\beta}}(\bm{\beta}^*)(\bm{\beta} - \bm{\beta}^*) \\
\end{align*}
If we set this to $0$ we further get:
$$\bm{H}_{\bm{\beta}}(\bm{\beta}^*) (\bm{\beta} - \bm{\beta}^*) = -\nabla_{\bm{\beta}} C(\bm{\beta}^*)$$
which leads to
$$\bm{\beta} - \bm{\beta}^* = - \bm{H}_{\bm{\beta}}(\bm{\beta}^*)^{-1} \nabla_{\bm{\beta}} C(\bm{\beta}^*)$$
and finally
$$\bm{\beta} = \bm{\beta}^* - \bm{H}_{\bm{\beta}}(\bm{\beta}^*)^{-1} \nabla_{\bm{\beta}} C(\bm{\beta}^*)$$

This tells us that the minimum of the cost-function, using any $\bm{\beta}^*$ is
given as above. The problem with this is that this is using the
Taylor-expansion, which is only reliable close to $\beta^*$ so chances are that
using this formula will not give us a global minimum. However it generally tends
to bring us closer to the minimum we are looking after. What we then usually do
is to start with some initial guess for the optimal $\bm{\beta}$, which we
denote $\bm{\beta}^{(0)}$. We then for $k=1, \dots$ let $\bm{\beta}^(k) =
    \bm{\beta}^{(k-1)} - \bm{H}_{\bm{\beta}}(\bm{\beta}^{(k-1)})^{-1}
    \nabla_{\bm{\beta}} C(\bm{\beta}^{(k-1)})$ all the way until we have convergence.

\subsection{Schedulers}
Something which often is problematic when using the Newton-Raphson method as
described over is that we need to estimate the Hessian matrix and also invert
it. First of all finding a way to calculate the second derivative can be a
difficulty in and of itself, but it may also be computationally expensive to do
so. Not only that we also need to invert this matrix which can also become very
computationally expensive, and perhaps also numerically unstable with higher
dimension hesse matrices. It's important to note that the Hessian matrix is a
square matrix with the same dimensions as $\bm{\beta}$ in both width and height.
Inverting this matrix then may not be a problem if we only have a few
parameters, but for example when dealing with logistic regression with lots of
features, or neural networks with lots of hidden nodes and layers this becomes
unviable. We therefore will use various methods to approximate this hessian matrix.

\subsubsection{Ordinary gradient descent}
Ordinary gradient descent is a very simple way of approximating the inverse of
the Hessian matrix. Here we simply approximate it with some numerical
learning-rate, which we denote $\lambda$. This way our algorithm becomes:
$$\bm{\beta}^(k) = \bm{\beta}^{(k-1)} - \lambda \nabla_{\bm{\beta}} C(\bm{\beta}^{(k-1)})$$
Often we may just keep this fixed, but we can also make it depend on $k$. For
example one may wish to start with a higher learning-rate and make this decrease
with the number of iterations. This method I will refer to as using time-decay.
When using gradient descent for training our model $\lambda$ then becomes a
hyperparameter in our model. Often this hyperparameter is rather important in
determining the fit of the model. We often want a rather high learning-rate as
this will lead to faster convergence towards some optimum, however if we make it
to big we may not get convergence at all. I will mainly throughout the code
simply set up a grid of learning-rates and fit to all of them, and simply choose
the learning-rate which gives the best performance, but there are countless
other more advanced way of tuning this parameter.

\subsubsection{Adam}

\bibliography{./sources.bib}

\end{document}